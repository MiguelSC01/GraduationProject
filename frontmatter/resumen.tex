\chapter*{Resumen}

El modelo discreto que intenta simular el proceso de vida de los glóbulos rojos propuesto por Eldestein en \cite{edelstein1988mathematical} es:

$$R(n+1)=(1-f)R(n)+M(n),$$
$$M(n+1)=\gamma \cdot f\cdot R(n),$$

en donde $R(n)$ representa la cantidad de glóbulos rojos en el día $n$ y $M(n)$ los glóbulos rojos producidos por la médula ósea. El parámetro $0\leq f \leq 1$ representa la fracción de glóbulos rojos filtrados por el bazo diariamente y $\gamma \geq 0$ representa la cantidad de RBC's producidos por la médula ósea por cada uno eliminado por el bazo. 

En el proyecto se hace un análisis matemático de este modelo base y se hacen tres simulaciones computacionales cambiando el valor de $\gamma$ (tomándolo mayor, menor o igual a 1). Cada simulación es analizada según las observaciones hechas desde el punto de vista matemático. 

Posteriormente se presentan tres variaciones del modelo siguiendo tres principales complicaciones médicas: una hemorragia leve (pérdida de sangre del $2\%$), una hemorragia grave (pérdida de sangre del $14\%$) y anemia renal (deficiencia de eritropoyetina). Para cada uno de los casos se presenta su tratamiento médico, su correspondiente simulación computacional y el análisis de esta. 

Para concluir se presentan nuevas variables a tener en cuenta a la hora de proponer el modelo y algunos cambios que se podrían hacer para modificar los resultados obtenidos. También se propone un trabajo a futuro para incluir al modelo la movilización de células madre para pacientes con cáncer.

