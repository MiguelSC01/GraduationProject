\chapter*{Resumen}

En el proyecto se hace un análisis matemático del modelo para determinar la cantidad de glóbulos rojos en el cuerpo humano propuesto por Edelstein-Keshet y se hacen tres simulaciones computacionales cambiando el valor de $\gamma$ (tomándolo mayor, menor o igual a 1). Cada simulación es analizada según las observaciones hechas desde el punto de vista matemático. 

Posteriormente se presentan tres variaciones del modelo siguiendo tres principales complicaciones médicas: una hemorragia leve (pérdida de sangre del $2\%$), una hemorragia grave (pérdida de sangre del $14\%$) y anemia renal (deficiencia de eritropoyetina). Para cada uno de los casos se presenta su tratamiento médico, su correspondiente simulación computacional y el análisis de esta. Para el caso de la anemia renal, se halla una dosis que permite evitar una pérdida constante de glóbulos rojos. 

Para concluir se presentan nuevas variables a tener en cuenta a la hora de proponer el modelo y algunos cambios que se podrían hacer para modificar los resultados obtenidos. También se propone un trabajo a futuro para incluir al modelo la movilización de células madre para pacientes con cáncer.

