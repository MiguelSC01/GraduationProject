\chapter{Solución de Sistemas de Ecuaciones en Diferencias Lineales Homogéneas de Segundo Grado}\label{chap:Apendice}

En el presente apéndice, se estudiarán las ecuaciones en diferencias lineales homogéneas de segundo grado con coeficientes constantes, los sistemas de estos y el mecanismo para obtener sus respectivas soluciones siguiendo lo expuesto por Banerjee en \cite{banerjee2021mathematical}.

\begin{definition}
Una \emph{ecuación en diferencias lineal homogénea de segundo grado con coeficientes constantes} es:
\begin{equation}\label{eq:eqlineal}
    c_0X(n)+c_1X(n-1)+c_2X(n-2)=0,
\end{equation}
en donde los $c_i$ son constantes para todo $i$.
\end{definition}

Para solucionar la ecuación \ref{eq:eqlineal}, considere inicialmente la ecuación en diferencias homogénea de primer grado
$$X(n+1)=\lambda\cdot X(n),$$
a esta se le construye la solución de la siguiente manera (conociendo el valor de $X(0)$):
$$X(1)=\lambda\cdot X(0),$$
$$X(2)=\lambda\cdot X(1)= \lambda\cdot (c\cdot X(0))=\lambda^2\cdot X(0)$$
$$\vdots$$
$$X(n)=\lambda^n X(0).$$ 
De esta manera, la solución general de la ecuación de primer orden es $X(n)=\lambda^n\cdot c$, en donde $c = X(0)$.

Ahora, para la de segundo grado se sigue este mismo razonamiento (conocido como supuesto de Euler) y se asume que la solución tiene la forma $c\cdot (\lambda)^n$ en donde $c \neq 0$. Reemplazando esta solución en la ecuación inicial, se obtiene
$$c_0[c\cdot (\lambda)^n]+c_1[c\cdot (\lambda)^{n-1}]+c_2[c\cdot (\lambda)^{n-2}]=0,$$
dividiendo por $c\cdot(\lambda)^{n-2}$ da cada adendo, se obtiene el \emph{polinomio característico} de la ecuación
$$c_0\cdot\lambda^2 + c_1\cdot \lambda +c_2=0.$$

Al ser un polinomio de segundo grado, sean $\lambda_1$, $\lambda_2$ sus soluciones (note que pueden ser iguales), entonces se llega al siguiente teorema:

\begin{theorem}
    Las soluciones a la ecuación en diferencias \ref{eq:eqlineal} están dadas por todas las combinaciones lineales de las $n$-ésimas potencias de las raíces $\lambda_1$, $\lambda_2$ del polinomio característico
    $$p(\lambda)=c_0\cdot\lambda^2 + c_1\cdot \lambda +c_2,$$
    es decir que el conjunto de soluciones de la ecuación en diferencias es
    $$S=\{k_1\cdot \lambda_1^n+k_2\cdot \lambda_2^n\;|\;k_i\in\mathbb{R}\}.$$

    En caso de que $\lambda_1=\lambda_2$, entonces las soluciones son combinaciones lineales de $\lambda_1$ y $n\cdot \lambda_1$.

    Si las raíces son complejas, entonces se toma como solución general
    $$S=\{r^n(k_1\cdot \cos(n\theta)+k_2\cdot \sin(n\theta))\;|\;l_i\in\mathbb{R}\},$$
    en donde $r,\theta$ son el radio y ángulo de la representación polar de las raíces (note que son conjugadas).
\end{theorem}

\textbf{Demostración:}

Para la demostración, solo se tomará el caso real de raíces distintas.

Sea $X(n)=k_1\cdot \lambda_1^n+k_2\cdot \lambda_2^n$, al reemplazarlo en la ecuación original, se tiene

$$c_0[k_1\cdot (\lambda_1)^n+k_2\cdot (\lambda_2)^n]+c_1[k_1\cdot (\lambda_1)^{n-1}+k_2\cdot (\lambda_2)^{n-1}]+c_2[k_1\cdot (\lambda_1)^{n-2}+k_2\cdot (\lambda_2)^{n-2}]=0$$
$$\implies (c_0[c_1\cdot \lambda_1^n]+c_1[c_1\cdot \lambda_1^{n-1}]+c_2[c_1\cdot \lambda_1^{n-2}])+(c_0[c_2\cdot \lambda_2^n]+c_1[c_2\cdot \lambda_2^{n-1}]+c_2[c_2\cdot \lambda_2^{n-2}])=0$$
$$\implies c_1\cdot\lambda_1^{n-2}p(\lambda_1)+c_2\cdot\lambda_2^{n-2}p(\lambda_2)=0$$
$$\implies c_1\cdot\lambda_1^{n-2}(0)+c_2\cdot\lambda_2^{n-2}(0)=0$$
$$\implies 0=0$$ \qed

Ahora, considere el sistema homogéneo de ecuaciones en diferencias

\begin{align}\label{eq:sisqeqdif}
    X(n+1) &= \alpha X(n) + \beta Y(n), \\
    Y(n+1) &=\gamma X(n) + \delta Y(n). \nonumber
\end{align}

Note que este se puede representar mediante la matriz 
$$Z_{n+1}=\begin{pmatrix}
    X(n+1) \\
    Y(n+1) 
    \end{pmatrix}=
    \begin{pmatrix}
    \alpha & \beta\\
    \gamma  & \delta 
    \end{pmatrix} \cdot 
    \begin{pmatrix}
    X(n) \\
    Y(n) \\
    \end{pmatrix}=
    \begin{pmatrix}
        \alpha & \beta\\
        \gamma  & \delta 
        \end{pmatrix} Z_n,$$

$$ A = \begin{pmatrix}
    \alpha & \beta\\
    \gamma  & \delta 
    \end{pmatrix} $$

Para solucionar este sistema, se van a utilizar los valores y vectores propios de la matriz $A$:

\begin{theorem}
    Asuma que los valores propios de la matriz $A$ son reales y distintos. Las soluciones del sistema homogéneo de ecuaciones en diferencias \ref{eq:sisqeqdif} están dadas por la fórmula
    $$Z_n = k_1 (\lambda_1)^n \mathbf{v_1}+k_2 (\lambda_2)^n \mathbf{v_2},$$
    en donde los $\lambda_i$ son los valores propios de la matriz $A$, $\mathbf{v_i}$ su respectivo vector propio asociado y $k_i$ una constante real.  
\end{theorem}

\textbf{Demostración:}

$$A\cdot Z_n = k_1(\lambda_1)^n(A\cdot \mathbf{v_1})+ k_2(\lambda_2)^n(A\cdot \mathbf{v_2})$$
$$=k_1(\lambda_1)^n(\lambda_1\mathbf{v_1})+ k_2(\lambda_2)^n(\lambda_2 \mathbf{v_2})$$
$$=k_1 (\lambda_1)^{n+1} \mathbf{v_1}+k_2 (\lambda_2)^{n+1} \mathbf{v_2}=Z_{n+1} \qed$$

Finalmente, tanto para el sistema como para la ecuación en una variable, hallar la solución específica dependerá de las condiciones iniciales $X(0)$, $X(1)$ para la ecuación y $X(0)$, $Y(0)$ para el sistema. Basta hacer los reemplazos apropiados y hallar los valores de $k_i$.
