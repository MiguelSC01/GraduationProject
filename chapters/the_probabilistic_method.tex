\chapter{The Probabilistic Method}\label{chap:litrev}

\section{Introduction}\label{sec:litrev:intro}

The probabilistic method is a powerful tool, with applications in Combinatorics, Graph Theory, Number Theory and Computer Science. It is a nonconstructive method that proves the existence of an object with a certain property by showing that the probability that a randomly chosen object has that property is greater than zero. This requires an appropiate sample space. \par

The method is best illustrated by an example: \par

\begin{definition}\label{def:tournament}
A \textit{tournament} is a directed graph $T$ on $n$ vertices such that for every pair of vertices $i, j \in V(T)$, exactly one of the edges $(i, j)$ or $(j, i)$ is in $E(T)$. \cite{alon2016probabilistic}
\end{definition}

\begin{figure}
    \centering
    \ctikzfig{tournament}
    \caption{A tournament on 4 vertices.}
    \label{fig:tournament}
\end{figure}

Section \ref{sec:litrev:theme1} discusses Theme 1. Section \ref{sec:litrev:theme2} discusses Theme 2....

\section{Union Bound}\label{sec:litrev:theme1}


\subsection{Example}\label{sec:litrev:theme1:A}

\subsection{Subtopic B}\label{sec:litrev:theme1:B}

\subsection{Subtopic C}\label{sec:litrev:theme1:C}

\section{Linearity of Expectation}\label{sec:litrev:theme2}

Etc. etc. \ref{def:tournament}

\section{Second Moment Method}

\section{Threshold Functions}

Let $n \in \N$ and $0 \leq p \leq 1$. The random graph $G(n, p)$ is a probability space over the set of graphs on $n$ labeled vertices determined by
\[\Pr[\{i, j\} \in G] = p\] 
with these events mutually independent \cite{alon2016probabilistic}. Given a graph theoretic property $A$, there is a probability that $G(n, p)$ satisfies $A$, which we write as $\Pr[G(n, p) \vDash A]$. 

\begin{definition}
    $r(n)$ is a threshold function for a graph theoretic property $A$ if 
    \begin{enumerate}
        \item When \(p(n) \in o(r(n)), \; \lim_{n \to \infty} \Pr[G(n, p(n)) \vDash A] = 0,\)
        \item When \(r(n) \in o(p(n)), \;  \lim_{n \to \infty} \Pr[G(n, p(n)) \vDash A] = 1,\) 
    \end{enumerate}
    or vice versa. \cite{alon2016probabilistic}
\end{definition}

We give an example of a threshold function which illustrates a common method for proving that a function is a threshold. \par

\subsubsection{Threshold function for having isolated vertices}

Let $G$ be a graph on $n$ labeled vertices. An isolated vertex of $G$ is a vertex which does not belong to any of the edges of $G$. Let $A$ be the property that $G$ contains an isolated vertex. We will prove that $\displaystyle{r(n) = \frac{\ln n}{n}}$ is a threshold for $A$. \par

For each vertex $i$ in $G$ define the variable 

\[X_i = 
\left\{
	\begin{array}{ll}
		1  & \mbox{if } i \text{ is an isolated vertex,} \\
		0 & \mbox{if } i \text{ is not an isolated vertex.}
	\end{array}
\right.
\]

Now, the probability that a vertex $i$ is isolated is $(1 - p)^{n - 1}$ since it is the probability that none of the other $n - 1$ vertices is connected to $i$. Let $X = \sum_{i = 1}^n X_i$, then the expected number of isolated vertices is
 \[E[X] = \sum_{i = 1}^{n} E[X_i] = \sum_{i = 1}^{n} \Pr[X_i] = n(1 - p)^{n - 1}.\]

Let $\displaystyle{p = k\frac{\ln n}{n}}$ for $k \in \R_{>0}$. Then
\begin{align*}
    \lim_{n \to \infty} E[X] &= \lim_{n \to \infty} n\left(1 - k\frac{\ln n}{n}\right)^{n - 1} \\
    &= ne^{-k\ln n} = n^{1 - k}.
\end{align*}

Therefore, $\lim_{n \to \infty} E[X] = 0$ if $k > 1$. Since \(E[X] \geq \Pr[X > 0],\) we conclude that \[\lim_{n \to \infty} \Pr[G(n, p) \vDash A] =  \lim_{n \to \infty} \Pr[X > 0] = 0.\] \par
Now, for $k < 1$, the fact that $\lim_{n \to \infty} E[X] = \infty$ is not enough to conclude that \(\lim_{n \to \infty} \Pr[G(n, p) \vDash A] = 1\). We have to use the second moment method. \par

\begin{theorem*}
    If $E[X] \to \infty$ and $\text{Var}[X] = o(E[X]^2)$, then $\lim_{n \to \infty} \Pr[X > 0] = 1$. \cite{alon2016probabilistic}
\end{theorem*}

\textbf{Proof. } We will prove that, in this case, $Var[X] = o(E[X]^2)$. First, 
\begin{align*}
    \sum_{i \neq j}E[X_iX_j] &= \sum_{i \neq j} \Pr[X_i = X_j = 1] \\
    &= n(n - 1)(1 - p)^{n -1}(1 - p)^{n - 2} \\ &= n(n - 1)(1 - p)^{2n - 3},
\end{align*}

for if $i$ is an isolated vertex, then there is no edge between $i$ and $j$ so we only have to account for the remaining $n - 2$ edges that contain $j$.  \par

Thus, since $\sum_{i = 1}^{n}E[X_i^2] =  \sum_{i = 1}^n E[X_i] = E[X]$ and $\lim_{n \to \infty} p = 0$,

\begin{align*}
    \lim_{n \to \infty} \frac{\text{Var}[X]}{E[X]^2} &= 
    \lim_{n \to \infty}\frac{E(X^2) - E[X]^2}{E[X]^2} = \lim_{n \to \infty} \frac{\sum_{i = 1}^n E[X_i^2] + \sum_{i \neq j}E[X_iX_j]}{E[X]^2} - 1 \\ &= 0 + \lim_{n \to \infty} \frac{ n(n - 1)(1 - p)^{2n - 3}}{n^2(1 - p)^{2n - 2}} - 1= \lim_{n \to \infty} \frac{1}{1 - p} - 1 = 0.\\
\end{align*} \par
We conclude that $\text{Var}[X] \in o(E[X]^2)$ and so, if $k < 1$, \[\lim_{n \to \infty} \Pr[G(n, p) \vDash A] = \lim_{n \to \infty} \Pr[X > 0] = 1.\] Therefore, $r(n) = \frac{\ln n}{n}$ is a threshold function for property $A$.
