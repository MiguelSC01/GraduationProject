\chapter{Introducción}\label{chap:intro}
Los glóbulos rojos (RBC's) constituyen alrededor del 50\% de la sangre en el cuerpo humano. Entre sus funciones se encuentra el transporte de oxígeno a los diferentes órganos y tejidos del cuerpo para su correcto funcionamiento, es por esto que son fundamentales para la vida humana, pues todos los órganos fallarían de no ser por su existencia. De esta manera, poder controlar y medir la cantidad de RBC's en el cuerpo es una tarea muy importante para la labor médica, pues en ciertas enfermedades o complicaciones puede ocurrir una disminución de la cantidad de estas células en el torrente sanguíneo y, por lo tanto, ciertas funciones vitales pueden fallar. (\cite{cosgrove2021hematopoiesis}\cite{hall2021guyton}\cite{schippel2023dynamics}\cite{winter2014molecular}\cite{higgins2015red})

Los modelos matemáticos con ecuaciones en diferencias (discretos) o ecuaciones diferenciales (continuos) se han transformado en los últimos años en componentes fundamentales para varios campos de la vida humana como la economía, la biología o medicina. Utilizando el modelo desarrollado del problema propuesto se pueden realizar simulaciones computacionales de este par poder así generar una predicción matemáticamente acertada de lo que se busque. (\cite{edelstein1988mathematical}\cite{liu2023research}\cite{murray2007mathematical})

La investigación buscará modelar y simular matemáticamente las dinámicas de los glóbulos rojos en el cuerpo humano de una forma discreta, es decir día a día. El objetivo principal del trabajo es analizar, en un ambiente matemático y médico, la homeostasis (equilibrio) del modelo y también modificarlo para incluir diferentes complicaciones físicas, como la anemia o hemorragias, que pueden perturbar la homeostasis del modelo propuesto, observar su comportamiento y buscar soluciones para volver a lograr la homeostasis, como la aplicación de eritropoyetina, de hierro o de transfusiones en el cuerpo humano. (\cite{bunn2013erythropoietin}\cite{heras2023anemia}\cite{portoles2021anemia}\cite{shrestha2016models}\cite{kirk1968mathematical})

El modelo base escogido está basado en el propuesto por Edelstein \cite{edelstein1988mathematical} en el segundo problema de la página 27 y el decimosexto de la página 33.

Sea $R=R(n)$ el número de glóbulos rojos circulando en el torrente sanguíneo en el día $n$, $M=M(n)$ el número de glóbulos rojos producidos por la médula ósea en el día $n$, $f>0$ la fracción de RBC's eliminada por el bazo y $\gamma>0$ la constante de producción, es decir la cantidad de glóbulos producida por cada glóbulo perdido. El modelo resultante es:
$$R(n+1)=(1-f)R(n)+M(n)$$
$$M(n+1)=\gamma \cdot f\cdot R(n)$$

A este modelo, y a los demás derivados de este al hacer las modificaciones adecuadas, se le hará el análisis matemático para encontrar la solución analítica, a través de los valores y vectores propios de la matriz que generan las ecuaciones, y se hallarán los equilibrios del modelo, determinando los valores de los parámetros para los cuales estos son estables o inestables. Sobre los modelos modificados, es decir con enfermedades o complicaciones fisiológicas y sus posibles curas, se presentará la explicación de los resultados obtenidos sobre las simulaciones y se intentará determinar la cura óptima para cada enfermedad y la cantidad a administrar para no perturbar la homeostasis. Para los medicamentos y las enfermedades, hay dos opciones: que sean de una sola vez (como hemorragias o medicamentos de una dosis) o multiples veces (multidosis y anemia).

Las conclusiones y los análisis presentados se espera que sean útiles para el campo médico, pues con estos se busca optimizar tiempo y fármacos, además de (lo más importante) lograr salvar vidas humanas.

La estructura del proyecto de grado es la siguiente:
\begin{itemize}
    \item El capítulo dos presenta la introducción médica de los glóbulos rojos, es decir su composición, sus funciones y su ciclo de vida.
    \item El capítulo tres presenta el análsis matemático del modelo base, así como las simulaciones computacionales de este y su respectivo análisis.
    \item El capítulo cuatro presenta variaciones del modelo nacidas de enfermedades, así como las simulaciones computacionales y su análisis.
    \item El capítulo cinco presenta las conclusiones del proyecto y el trabajo a futuro que se puede hacer a partir de la investigación.
\end{itemize}
