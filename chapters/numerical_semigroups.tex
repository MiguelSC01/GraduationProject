\chapter{Numerical Semigroups}\label{chap:probmet}

\section{Introduction}\label{sec:probmet:intro}

Some contextural overview of what we are going to discuss. Include the first example in the book. 

Section \ref{sec:litrev:theme1} discusses Theme 1. Section \ref{sec:litrev:theme2} discusses Theme 2....

A \textit{numerical semigroup} is a subset $S \subseteq \NN_{0}$ which is closed under addition, i.e. $a, b \in S$ implies $a + b \in S$. For instance, $\NN_0$, $\NN_0 \setminus \{0\}$, $2\NN_0$ are all numerical semigroups, but $\NN_0 \setminus \{2\}$ is not. Some literature requires that a semigroup has a finite complement in $\Z_{\geq 0}$ \cite{chapman2020beyond}, but we prefer the more general definition. 


\section{Invariants}\label{sec:probmet:theme1}


\subsection{Subtopic A}\label{sec:probmet:theme1:A}

\subsection{Subtopic B}\label{sec:probmet:theme1:B}

\subsection{Subtopic C}\label{sec:probmet:theme1:C}

\section{Wilf's Conjecture}\label{sec:probmet:theme2}

Etc. etc.
