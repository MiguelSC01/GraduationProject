\chapter{Numerical Semigroups}\label{chap:smgs}

\section{Introduction}\label{sec:smgps:intro}

So far we have only discussed graphs. In this chapter, we will introduce a new object which has a different structure, but for which the probabilistic method can be used to prove results. \par

\begin{definition} \cite{rosales2009numerical}
    A \textit{numerical semigroup} is a subset $S \subseteq \NN_{0}$ for which 
    \begin{enumerate}
        \item $0 \in S$,
        \item $S$ is closed under addition, i.e. $a, b \in S$ implies $a + b \in S$, and
        \item $S$ has finite complement in $\NN_{0}$.
    \end{enumerate}
\end{definition}

Examples of numerical semigroups include $\NN_{0}$ and $\NN_{0} \setminus \{1\}$. Subsets of $\NN_0$ which are not numerical semigroups include the set of even numbers, any finite set and $\NN_0 \setminus \{2\}$. \par

\begin{example}\label{ex:smgps:mcnugget}
    The \textit{McNugget Semigroup} is the set of all non-negative integers which can be expressed as a sum of non-negative multiples of 6, 9 and 20. 
\end{example}

Suppose you are in the United Kingdom and you wish to order 43 McNuggets. The cashier will hesitate for a while before telling you that they do not sell 43 McNuggets, since there is no combination of boxes of 6, 9 and 20 McNuggets which add up to 43. However, if you order 44 McNuggets, one possibility is that you will receive one box of 20 McNuggets, two boxes of 9 McNuggets and one box of 6 McNuggets. This is because 44 can be expressed as a sum of non-negative multiples of 6, 9 and 20, namely $44 = 2 \cdot 20 + 2 \cdot 9 + 6$ \cite{youtube}. In general, if you order more than 43 McNuggets, you will receive your order, since any integer greater than or equal to 43 can be expressed as a sum of non-negative multiples of 6, 9 and 20. \par

Let us see why the McNugget Semigroup is a numerical semigroup. First, we note that $0$ can be expressed as a sum of non-negative multiples of 6, 9 and 20, namely $0 = 0 \cdot 6 + 0 \cdot 9 + 0 \cdot 20$. Next, we note that if $a$ and $b$ can be expressed as a sum of non-negative multiples of 6, 9 and 20, then so can $a + b$. Finally, we note that the complement of the McNugget Semigroup in $\NN_0$ is finite, since 
\begin{align*}
    44 &= 2 \cdot 20 + 2 \cdot 9 + 6, &45 &= 5 \cdot 9,\\
    46 &= 2 \cdot 20 + 6, &47 &= 20 + 3 \cdot 9,\\
    48 &= 8 \cdot 6, & 49&= 2 \cdot 20 + 9. 
\end{align*}
And every integer greater than 49 can be expressed as a sum of one of these numbers plus a multiple of 6. \par
The McNugget semigroup is an example of a numerical semigroup which is \textit{finitely generated}. This means that there exists a finite set $A = \{a_1, \ldots, a_n\}$ such that $S = \langle A \rangle$, where 
\[\langle A \rangle = \{c_1a_1 + \cdots + c_na_n : c_1, \ldots, c_n \in \NN\}.\]  
 In general,
\begin{theorem}\label{thm:smgps:fin_gen}
    All numerical semigroups are finitely generated.
\end{theorem}

\textbf{Proof. } Let $S$ be a numerical semigroup. Let $m$ be the first non-zero element of $S$. Let $b_i$ be the first element of $S$ such that \(b_i \equiv i \mod m\), which exists since $S$ has a finite complement in $\NN$. Let $A = \{m, b_{1}, \ldots, b_{m - 1}\}$. Then $S = \langle A \rangle$, since every non-zero element of $S$ can be expressed as a sum of an element of $A$ plus a non-negative multiple of $m$. \qed \par

Also, note that $\text{gcd}(\{6, 9, 20\}) =  1$ and, in general,

\begin{theorem}\label{thm:smgps:gcd}
    Let $A \subseteq \NN$ be a non-empty finite set. Then $\langle A \rangle$ is a numerical semigroup if and only if $\text{gcd}(A) = 1$.    
\end{theorem}

\textbf{Proof. } Let $A = \{a_{1} \ldots, a_{n}\}$, where $a_1$ is the first non-zero number in $S = \langle A \rangle$. Note that $a_1$ is in $A$, since it cannot be expressed as the sum of non-negative multiples of other elements in $S$. \par
If $\text{gcd}(A) = d > 1$, then every element in $S$ is divisible by $d$ and so there are infinitely many numbers in $\NN$ which are not in $S$. \par
Now, suppose that $\text{gcd}(A) = 1$. If $a_1 = 1$, then $S = \NN$ is a numerical semigroup. Suppose that $a_1 > 1$. By definition of generating set, $0 \in S$ and S is closed under addtion. Since $\text{gcd}(A) = 1$ then there exist $\lambda_1, \ldots, \lambda_n \in \ZZ$ such that 
\[\lambda_1a_1 + \ldots + \lambda_na_n = 1.\]
Then, 
\[
    k := \sum_{i = 1}^n \lambda_ia_i + a_1\sum_{i = 1}^n|\lambda_i|a_i \equiv 1 \mod a_1,
\]
and $k$ is a non-negative sum of multiples of elements of $A$: 
\[k = \sum_{i = 1}^n (\lambda_i + |\lambda_i|a_1)a_i,\]
and so $k \in S$. Thus, for $0 \leq j < a_1$, $jk \in S$ and $jk$ is congruent with $j$ modulo $a_1$. Therefore, every number greater than $(a_1 - 1)k$ belongs to $S$, since it can be expressed as the sum of a multiple of $k$ plus a multiple of $a_1$.  This means that the complement of $S$ in $\NN$ is finite and so $S$ is a numerical semigroup. \qed \par

The McNugget semigroup and the proofs of the previous theorems motivates the following numerical semigroup invariants. \par 

\section{Invariants}\label{sec:smgps:invariants}

Let $S$ be a numerical semigroup. \par

\begin{definition}\label{def:smgps:multiplicity}
    The \textit{multiplicity} of $S$, denoted by $m(S)$, is the smallest non-zero element of $S$.
\end{definition}

For instance, the multiplicity of the McNugget Semigroup is 6. \par

Let $A$ and $B$ be non-empty finite subsets of $\NN$. Then we denote by $A + B$ the set \[\{a + b : a \in A, b \in B\}.\] 

\begin{theorem}\label{thm:smgps:minimal_generating_set}
     There exists a unique minimal generating set $A$ with $S = \langle A \rangle$.
\end{theorem}

\textbf{Proof. } Let $A = S \setminus(S + S)$. This means that every element in $A$ is not the sum of two elements in $S$. First we prove that $A$ generates $S$. Note that $A$ generates 0. Suppose that $s\in S \setminus A$. Then $s = a + b$, such that $a$ and $b$ are in $S$ and $a, b < s$. If we proceed recursively, in a finite number of steps we can express $s$ as a sum of elements of $A$. \par

Now, we show that $A$ is minimal. If $S = \langle A' \rangle$, then for $a \in A$, if $a$ is a sum of non-negative multiples of elements of $A'$, then, since $a$ is not the sum of two elements in $S$, $a$ must be an element of $A'$. \qed \par

This motivates \par

\begin{definition}\label{def:smgps:embedding_dim}
    The \textit{embedding dimension} of $S$, denoted by $e(S)$, is the cardinality of the minimal generating set of $S$.
\end{definition}

Theorem \ref{thm:smgps:minimal_generating_set} together with the proof of Theorem \ref{thm:smgps:fin_gen} show that $e(S) \leq m(S)$. \par

Let $n$ be a non-zero element of $S$. \par

\begin{definition}\label{def:smgps:aperyset}
    The \textit{Ap\'ery set} of $n$ in $S$ is the set
\end{definition}
\[\text{Ap}(S, n) = \{s \in S : s - n \notin S\}\]

For instance, $\text{Ap}(\langle 6, 9, 20 \rangle, 6) = \{0, 9, 20, 29, \}$. \par

\begin{definition}\label{def:smgps:frobeniusnum}
    The \textit{Frobenius number} of $S$, denoted by $F(S)$, is the largest element of $(\NN \cup \{-1\}) \setminus S$.
\end{definition}

The Frobenius number of the McNugget semigroup is 43. \par


\begin{definition}\label{def:smgps:genus}
\end{definition}


\section{Wilf's Conjecture}\label{sec:smgps:theme2}

Etc. etc.
