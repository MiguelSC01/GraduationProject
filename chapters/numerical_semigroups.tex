\chapter{Numerical Semigroups}\label{chap:probmet}

\section{Introduction}\label{sec:smgps:intro}

So far we have only discussed graphs. In this chapter, we will introduce a new object which has a different structure, but for which the probabilistic method can be used to prove results. \par

\begin{definition} \cite{rosales2009numerical}
    A \textit{numerical semigroup} is a subset $S \subseteq \NN_{0}$ for which 
    \begin{enumerate}
        \item $0 \in S$,
        \item $S$ is closed under addition, i.e. $a, b \in S$ implies $a + b \in S$, and
        \item $S$ has finite complement in $\NN_{0}$.
    \end{enumerate}
\end{definition}

Examples of numerical semigroups include $\NN_{0}$, 

\section{Invariants}\label{sec:smgps:theme1}

\begin{definition}[Multiplicity]
    
\end{definition}



\begin{definition}[Embedding Dimension]
\end{definition}

\begin{definition}[Apéry Set]
\end{definition}

\begin{definition}[Frobenius Number]

\end{definition}


\begin{definition}[Genus]

\end{definition}

\section{Wilf's Conjecture}\label{sec:smgps:theme2}

Etc. etc.
