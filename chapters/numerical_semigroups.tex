\chapter{Numerical Semigroups}\label{chap:smgs}

\section{Introduction}\label{sec:smgps:intro}

So far we have only discussed graphs. In this chapter, we will introduce a new object which has a different structure, but for which the probabilistic method can be used to prove results. \par

\begin{definition} \cite{rosales2009numerical}
    A \textit{numerical semigroup} is a subset $S \subseteq \NN_{0}$ for which 
    \begin{enumerate}
        \item $0 \in S$,
        \item $S$ is closed under addition, i.e. $a, b \in S$ implies $a + b \in S$, and
        \item $S$ has finite complement in $\NN_{0}$.
    \end{enumerate}
\end{definition}

Examples of numerical semigroups include $\NN_{0}$ and $\NN_{0} \setminus \{1\}$. Subsets of $\NN_0$ which are not numerical semigroups include the set of even numbers, any finite set and $\NN_0 \setminus \{2\}$. \par

\begin{example}\label{ex:smgps:mcnugget}
    The \textit{McNugget Semigroup} is the set of all non-negative integers which can be expressed as a sum of non-negative multiples of 6, 9 and 20. 
\end{example}

Suppose you are in the United Kingdom and you wish to order 43 McNuggets. The cashier will hesitate for a while before telling you that they do not sell 43 McNuggets. However, if you order 44 McNuggets, you will receive 2 boxes of 20 McNuggets and 2 boxes of 2 McNuggets. This is because 44 can be expressed as a sum of non-negative multiples of 6, 9 and 20, namely $44 = 2 \cdot 6 + 2 \cdot 9 + 2 \cdot 20$ \cite{youtube}. 

Let us see why the McNugget Semigroup is a numerical semigroup. First, we note that $0$ can be expressed as a sum of non-negative multiples of 6, 9 and 20, namely $0 = 0 \cdot 6 + 0 \cdot 9 + 0 \cdot 20$. Next, we note that if $a$ and $b$ can be expressed as a sum of non-negative multiples of 6, 9 and 20, then so can $a + b$. Finally, we note that the complement of the McNugget Semigroup in $\NN_0$ is finite, since any integer greater than or equal to 43 can be expressed as a sum of non-negative multiples of 6, 9 and 20. \par

\section{Invariants}\label{sec:smgps:theme1}

\begin{definition}[Multiplicity]
    
\end{definition}



\begin{definition}[Embedding Dimension]
\end{definition}

\begin{definition}[Apéry Set]
\end{definition}

\begin{definition}[Frobenius Number]

\end{definition}


\begin{definition}[Genus]

\end{definition}

\section{Wilf's Conjecture}\label{sec:smgps:theme2}

Etc. etc.
