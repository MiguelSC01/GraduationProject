% All contribution chapters should follow a similar structure, with a
% mini-introduction and overview at the beginning and a conclusion at the
% end bookmarking a structured presentation of the contribution. This can be
% largely based on your publications.

\chapter{Modelo Base}\label{chap:modelo}
\section{Introducción}\label{sec:modelo:intro}
Ya tieniendo en cuenta las dinámicas de los glóbulos rojos en el cuerpo humano, se procederá a analizar y simular el modelo matemático de estas propuesto por Eldestein \cite{edelstein2005} utlizando las técnicas presentadas en \ref{chap:Apendice}. 

Es importante tener en cuenta que el modelo que sea que se analice debe tener concordancia con la vida real en cuanto a las variables y a las constantes utilizadas. La homeostasis es fundamental en todo proceso biológio y es por eso que el modelo y las simulaciones deben reflejar correctamente el equilibrio del sistema. Adicionalmente, es clave considerar que es imposible lograr una formulación matemática que este totalmente en concordancia con todos los procesos biológicos que suceden dentro del cuerpo humano, pues se debería tener en consideración una enorme cantidad de variables, de actores y de estado que se salen del propósito básico de simplificar el problema. De esta manera, los modelos propuestos son un acercamiento aproximado a la realidad del problema.

\section{Presentación del Modelo}\label{sec:modelo:presentacion}
Considerando la elemincación de glóbulos rojos del bazo y su producción por la medula ósea para mantener la homeostasis y el diagrama de cajas de la figura \ref{sec:RBC:fig:VidaRBC}, el modelo propuesto por Eldestein es el siguiente:
$$R(n+1)=(1-f)R(n)+M(n),$$
$$M(n+1)=\gamma \cdot f\cdot R(n),$$

en donde $R=R(n)$ representa la cantidad de glóbulos rojos por mililitro de sangre y $M=M(n)$ la cantidad de glóbulos rojos por mililitro de sangre que prodeuce la médula ósea. $\gamma>0$ representa la cantidad de glóbulos rojos producida por cada uno perdido y $0\leq f \leq 1$ el porcentaje de glóbulos rojos por mililitro de sangre que elimina el bazo. Ni $\gamma$ ni $f$ tienen unidades dadas estas definiciones.

De esta manera, el modelo se puede interpretar de la siguiente manera: la cantidad de RBC's en el día $n+1$ ($R(n+1)$) es el porcentaje restante del día $n$ más lo que haya producido la médula ósea en el día $n$; mientras que la cantidad de eritrocitos producidos por la médula ósea en el día $n+1$ ($M(n+1)$) es un múltiplo de la cantidad de glóbulos rojos eliminada en el día $n$. Este modelo es una simplificación bastante acertada de lo visto en \ref{sec:RBC:fig:CicloSangre} en cuanto a la relación entre producción y eliminación de RBC's; es claro que para ser aún más acertados se podrían agregar muchas otras variables como el funcionamiento de los riñones para la producción de eritropyetina, el proceso de absorción del hierro o la edad y estado de salud del sujeto analizado, pero estos complican innecesariamente el modelo y su análisis.

\section{Análisis del Modelo}\label{sec:modelo:analisis}
Ahora se procederá a analizar las soluciones del modelo a través de los métodos de ecuaciones en diferencias y de la matriz resultante del modelo.
\subsection{Método mediante Ecuaciones en Diferencias}

Nótese que las ecuaciones del modelo se pueden reducir a una única ecuación en términos de $R(n)$ de la siguiente manera:
$$M(n)=\gamma f R(n-1)$$
$$\implies R(n+1)=(1-f)R(n)+\gamma f R(n-1)$$
$$\implies R(n+2)=(1-f)R(n+1)+\gamma f R(n)$$
$$\implies R(n+2)-(1-f)R(n+1)-\gamma f R(n)=0.$$

Esta ecuación se puede tomar como una ecuación en diferencia homogenea lineal de grado 2, por lo que se pueden hallar sus soluciones al encontrar las raíces del polinomio característico de la ecuación:
$$p(\lambda)=\lambda^2-(1-f)\lambda-\gamma f$$
$$\implies \lambda_{1,2}=\dfrac{(1-f)\pm\sqrt{(1-f)^2+4\gamma f}}{2}.$$

Dado que $0\leq f \leq 1$ y $\gamma > 0$, entonces el radicando será positivo, por lo que los valores $\lambda_{1,2}$ son números reales, y de esta manera la solución general de la ecuación en diferencias es de la forma:
$$R(n)=c_1 \lambda_1^n+ c_2 \lambda_2^n,$$
en donde $c_{1,2}$ dependerán de las condiciones iniciales que se utilicen.

\subsection{Método mediante Matrices}

La matriz resultante del modelo es:
$$X_{n+1}=\begin{pmatrix}
    R(n+1) \\
    M(n+1) 
    \end{pmatrix}=
    \begin{pmatrix}
    1-f & 1\\
    \gamma f & 0 
    \end{pmatrix} \cdot 
    \begin{pmatrix}
    R(n) \\
    M(n) \\
    \end{pmatrix}=
    \begin{pmatrix}
        1-f & 1\\
        \gamma f & 0 
        \end{pmatrix} X_n,$$

por lo que para el ánalisis de soluciones se debe estudiar la matriz
$$A=\begin{pmatrix}
    1-f & 1\\
    \gamma f & 0 
    \end{pmatrix}.$$

Los autovalores de $A$ se obtienen al calcular las solciones $\lambda_{1,2}$ del determinante de $A-\lambda I$:
$$det(A-\lambda I) = \begin{vmatrix}
    1-f-\lambda & 1\\
    \gamma f & -\lambda 
    \end{vmatrix} = -\lambda(1-f-\lambda)-\gamma f$$

$$=\lambda^2-(1-f)\lambda-\gamma f,$$

utilizando la fórmula cuadrática, se obtiene
$$\lambda_{1,2}=\dfrac{(1-f)\pm\sqrt{(1-f)^2+4\gamma f}}{2}.$$

Los vectores propios correspondientes a estos valores propios son los vectores $\mathbf{v_{1,2}}$ que solucionan la ecuación
$$(A-\lambda_i I)\mathbf{v_i}=0, \:\:\: i=1,2,$$

y de esta manera la solución general del problema estará dada por

$$\begin{pmatrix}
    R(n) \\
    M(n) 
    \end{pmatrix}= (\lambda_1)^n \mathbf{v_1}+(\lambda_2)^n \mathbf{v_2}.$$


Nótese que ambos métodos, al arrojar el mismo polinomio característico, obtienen la misma solución general del sistema, esto implica que el modelo propuesto está bien fundamentado matemáticamente.

\section{Derivación de Parámetros y Análisis}\label{sec:modelo:parametros}
Para poder pasar a valores específicos de las soluciones encontradas, es necesario dar valores numéricos a los parámetros $\lambda$ y $f$, pues el modelo depende de estos y, junto a las condiciones iniciales, determinan las soluciones del problema. Para hallar estos parámetros, se utilizará la información brindada en \ref{sec:RBC:vida} sobre la producción de glóbulos rojos a través de la médula ósea y de su eliminación por medio del bazo.

Inicialmente, es necesario declarar las condiciones iniciales del problema. Dado que las solcuines encontradas dependen, en el caso de ecuaciones en diferencia, de dos constantes o, en el caso de matrices, de dos vectores, es necesario que hayan dos condiciones iniciales del problema: $R(0)$ y $R(1)$ o $X_0$. $R(0)$ representa la cantidad de glóbulos rojos al momento inicial de la medida, que se tomará como 125 trillones ($25\times 10^{12}$). Para el caso de la matriz, se puede tomar que $R(0)$ se mantiene como en el caso anterior y tomar $M(0)=0$, pues el instante que se inicia la medición, el cuerpo no ha generado nuevos glóbulos rojos. $M(0)$, como se verá más adelante en las simulaciones, es un valor muy interesante a tener en cuenta y que modifica relativamente los resultados obtenidos, pues al considerar o no que la médula ósea ha producido RBC's al momento de la primera medición, $R(1)$ cambiará.

Ahora, determinar $f$ es relativamente sencillo, pues es el porcentaje diario de RBC's que elimina el cuerpo. Contando con que la cantidad promedio es de eritrocitos es de $25\times 10^{12}$ y que diariamente mueren $208\times 10^{9}$, entonces el porcentaje de glóbulos rojos elminados diariamente por el bazo es del $0.832\%$, es decir que se puede tomar $f=0.00832$. 

Determinar el valor biológico de $\gamma$ es mucho más complicado, pues no ha sido posible determinar la cantidad de glóbulos rojos que produce una célula madre durante la eritropoyesis. Para hallar un valor adecuado de $\gamma$, se hará el análisis matemático y de las simulaciones con los valores ya establecidos. En todo caso, se puede hace el análisis para los diferentes casos:
\begin{enumerate}
    \item  $\gamma<1$: Este caso implica que el cuerpo produce una menor cantidad de eritrocitos respecto a la que elimina, por lo que los resultados deberían mostrar una caída en la cantidad de glóbulos rojos.
    \item $\gamma=1$: Este caso implica que el cuerpo produce la misma cantidad de eritrocitos respecto a la que elimina, por lo que los resultados deberían mostrar una estabilidad en la cantidad de glóbulos rojos.
    \item  $\gamma>1$: Este caso implica que el cuerpo produce una mayor cantidad de eritrocitos respecto a la que elimina, por lo que los resultados deberían mostrar un crecimiento en la cantidad de glóbulos rojos.
\end{enumerate}

En el caso esperado, es decir en el que el cuerpo alcanza la homeostasis, el valor deseado de $\gamma$ es $1$, pues de esta manera el cuerpo no pierde células, ya que todas aquellas que mueren son producidas. En todo caso, las simulaciones que se hagan utilizarán variaciones en este parámetro.

Como ya se ha visto en \ref{sec:modelo:analisis}, las soluciones del problema dependen de las constantes y de los valores iniciales, a continuación se calcularán las soluciones bajo los valores derivados para cada parámetro: $\gamma = 1$, $f = 0.00832$, $R(0)=25\times 10^{12}$, $M(0)=208\times 10^{9}$ 

\begin{itemize}
    \item Para el caso de ecuaciones en diferencias, se obtiene que las soluciones del polinomio carácterístico de la ecuación son:
        $$\lambda_{1,2}=\dfrac{(1-f)\pm\sqrt{(1-f)^2+4\gamma f}}{2}$$
        $$=\dfrac{(1-0.00832)\pm \sqrt{(1-0.00832)^2+4(0.00832)}}{2}$$
        $$\implies \lambda_1 = 1,\;\;\; \lambda_2 = -0.00832=-f.$$
        Y así, la solución general está dada por 
        $$R(n)=c_1+c_2(-0.00832)^n$$
        $$R(0)=25\times 10^{12}=c_1+c_2$$
        $$R(1)=(1-f)R(0)+M(0)=(1-0.00832)25 \times 10^{12}+208\times 10^9$$
        $$=c_1-0.00832\cdot c_2,$$
        resolviendo el sistema $2\times 2$ conformado por $c_{1,2}$ se obtienen los resultados $c_1=25\times 10^{12}$, $c_2 = 0$. Lo que quiere decir que la solución del problema, dadas las condiciones iniciales es:
        $$R(n)=25\times 10^{2},$$
        como es esperado, pues obtener una solución constante implica que el cuerpo logra mantener la homeostasis, es decir que en ningún momento hay pérdias o ganancias de RBC's en el cuerpo.
    \item En el caso de utilizar la matriz que representa el modelo, los valores propios hallados son
        $$\lambda_1 = 1, \;\;\; \lambda_2 = -0.00832=-f$$


\end{itemize}

\section{Simulaciones}\label{sec:modelo:simulaciones}
A continuación se presentarán las simulaciones computacionales del modelo y su respectivo análisis 
