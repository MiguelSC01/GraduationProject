% All contribution chapters should follow a similar structure, with a
% mini-introduction and overview at the beginning and a conclusion at the
% end bookmarking a structured presentation of the contribution. This can be
% largely based on your publications.

\chapter{Random Numerical Semigroups}\label{chap:randnumsems}

\section{Box Model}\label{sec:randomsmpgs:intro}

\begin{algorithm}
\caption{An algorithm with caption}\label{alg:cap}
\begin{algorithmic}
\Require $n \geq 0$
\Ensure $y = x^n$
\State $y \gets 1$
\State $X \gets x$
\State $N \gets n$
\While{$N \neq 0$}
\If{$N$ is even}
    \State $X \gets X \times X$
    \State $N \gets \frac{N}{2}$  \Comment{This is a comment}
\ElsIf{$N$ is odd}
    \State $y \gets y \times X$
    \State $N \gets N - 1$
\EndIf
\EndWhile
\end{algorithmic}
\end{algorithm}

\subsection{Expected Frobenius number}\label{sec:randomsmpgs:expectedfrob}

\section{ER-type model}

We generate a random numerical semigroup with a model similar to the Erdös-Rényi model for random graphs. 

\begin{definition}
    For $p \in [0, 1]$ and $M \in \NN$, a random numerical semigroup $S(M, p)$ is a probability space over the set of semigroups $S = \langle\mathcal{A}\rangle$ with $\mathcal{A} \subseteq \{1,...,M\}$, determined by
    \[\Pr[n \in \mathcal{A}] = p,\]
    with these events mutually independent.
\end{definition}

\subsection{Behavior of the ER-type model}

\section{Downward model}\label{sec:contrib1:theme2}

Etc. etc.



\section{Conclusion}
