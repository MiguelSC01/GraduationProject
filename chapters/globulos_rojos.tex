\chapter{Los glóbulos rojos en el cuerpo humano}\label{chap:RBC}
\section{Introducción}\label{sec:RBC:intro}
Este capítulo estará basado en los textos de Schippel \cite{schippel2023dynamics}, Hall \cite{hall2021guyton} y Thiagarajan \cite{thiagarajan2021red}. Se evidenciarán las funciones y dinámicas de los glóbulos rojos en el cuerpo humano desde una perspectiva completamente médica.

La hematología es el estudío biológico de la sangre, de sus componentes y de los desordenes que puede tener que afecten directamente el cuerpo humano. El estudio de la sangre es muy importante para la medicina ya que es una de las sustancias más abundantes del cuerpo y que tiene funciones muy importantes, como oxigenar todas las células para garantizar su correcto funcionamiento.

\section{Composición de glóbulos rojos}\label{sec:RBC:Composicion}
Los glóbulos rojos (RBC's) o eritrocitos son las células principales del sistema circulatorio en el cuerpo humano. Estos constituyen alrdedor del 50\% de la sangre, concentrando entre cuatro y seis millones de células por centímetro cúbico de sangre, y son fundamentales para la supervivencia de la especie, pues son los encargados de enviar óxigeno desde los pulmones hasta los diferentes órganos, así como otras sutancias importantes tales como aminoácidos y ácidos grasos; adicionalmente, también son los encargados de transportar ciertas sustancias nocivas para el cuerpo, como lo puede ser el dióxido de carbono, a los respectivos órganos que se encargan de eliminarlas.

Los eritrocitos son células sin núcleo con forma de disco bicóncavo con un radio aproximado de 4 micrómetros que son muy flexibles dado que tienen un exceso de membrana celular. La principal sutancia que contienen los glóbulos rojos es la hemoglobina, una proteína hecha principalmente de hierro que es la encargada del transporte del oxígeno y del dióxido de carbono, esta proteína es la que le otorga el color rojizo a la sangre humana. De esta manera, el hierro es una sutancia fundamental para la producción de los glóbulos rojos, y un nivel bajo de este, así como de vitamina B y ácido fólico, puede perjudicar gravemente las dinámicas de los eritrocitos.

Entre otras sutancias que se encuentran dentro de los RBC's, hay enzimas en su citoplasma que permiten flexibilidad, transporte de iones, estabilización de la hemoglobina y logran impedir la oxidación de otras proteínas.

\section{Proceso de vida de los glóbulos rojos}\label{sec:RBC:vida}

El origen inicial de los RBC's se encuentra en las células madre hematopoyéticas ubicadas en la médula ósea mediante el proceso de la eritropoyesis. Al detectar cantidades bajas de oxígeno en la sangre, los riñones se encargan de sintetizar la eritropoyetina (EPO), la hormona clave en la producción de estas células, para informarle al cerebro que debe producir más glóbulos rojos. 

Las células madre hematopoyéticas son las que producen todas las células necerias para la sangre. Al recibir el comando de la eritropoyetina, estas inician un proceso de transformación pasando a ser proeritroblastos, que al ser ya maduros, terminan multiplicándose y perdiendo el núcleo para convertirse finalmente en RBC's. Cada segundo, entre dos y tres millones de glóbulos rojos terminan su proceso de maduración y son colocados en el torrente sanguíneo para cumplir sus funciones.

Durante un periodo aproximado entre 90 y 120 días, unos tres o cuatro meses, los nuevos glóbulos rojos completan el ciclo cardiaco gracias a los bombeos del corazón, viajando de los pulmones, para recibir el óxigeno, hasta todos lo órganos, para entregarles el óxigeno y recibir el dióxido de carbono, como se puede evidenciar en la figura 2.1.

\begin{figure}[H]
    \centering
    \includegraphics[scale=0.2]{figures/CicloSangre.jpg}
    \caption{El ciclo que cumple la sangre en el cuerpo humano.}
    \label{sec:RBC:fig:CicloSangre}
\end{figure}

Al cumplir su vida útil, la membrana de los glóbulos rojos está debilitada y su tamaño reducido, por lo que pueden ser filtrados por el bazo y elminados por el cuerpo. También es posible que, al ser más frágiles de lo normal, los RBC's exploten dentro del torrente sanguíneo, pero las células macrófagas de la sangre son capaces de digerirlos rápidamente.

Contando con todo el ciclo expuesto, un diagrama de cajas, basado en aquel que se presenta en \cite{kirk1968mathematical}, que puede mostrar claramente el ciclo de vida de los glóbulos rojos es el siguiente:
\begin{figure}[H]
    \centering
    \includegraphics[scale=0.3]{figures/VidaRBC.jpeg}
    \caption{El proceso de producción y eliminación de glóbulos rojos.}
    \label{sec:RBC:fig:VidaRBC}
\end{figure}

En donde se desestima la transformación de células madre en ótro tipo de células del cuerpo, fuera de la mitosis para autorreproducirse, y la línea punteada implica que es bajo la acción de la eritropoyetina que las células madre reciben la señal de iniciar la eritropoyesis.

\section{Homeostasis}\label{sec:RBC:homeostasis}

Bioloógicamente, la homeostasis se define como la propiedad que tienen los organismos vivos de mantener un equilibrio estable al compensar pérdidas y ganancias. De esta manera, la hematología también es la encargada de estudiar la homeostasis de la sangre y de los glóbulos rojos en el cuerpo humano. Diferentes enfermades o percances como la anemia, definida como el déficit de glóbulos rojos o de hemoglobina en la sangre, o las hemorragias deben de ser tenidas en cuenta para generar y simular un buen modelo matemático de las dinámicas de los glóbulos rojos en el cuerpo humano.
