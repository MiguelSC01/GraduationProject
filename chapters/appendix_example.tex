\chapter{Useful Bounds}

We include some bounds that are useful in the proofs of the main results. By Stirling's Formula, we have that
\begin{equation}\label{ap:eq:stirling}
    k! \sim \sqrt{2\pi k}\left(\frac{k}{e}\right)^k.
\end{equation}

\begin{proposition}\label{ap:prop:upperbinom}
    $\binom{n}{k} \leq \left(\frac{en}{k}\right)^k$ for $1 \leq k \leq n$. 
\end{proposition}

\textbf{Proof. } Using (\ref{ap:eq:stirling}), we have that, for $k \geq 1$, 
\[k! \geq \left(\frac{k}{e}\right)^k.\]
Then
\[\binom{n}{k} \leq \frac{n^k}{\left(\frac{k}{e}\right)^k} = \left(\frac{en}{k}\right)^k. \qed\]

\begin{proposition}\label{ap:prop:lowebinom}
    $\left(\frac{n}{k}\right)^k \geq \binom{n}{k}$ for $1 \leq k \leq n$.
\end{proposition}

\textbf{Proof. }
\[\binom{n}{k} = \prod_{i = 0}^{k - 1} \frac{n - i}{k - i} \geq \left(\frac{n}{k}\right)^k. \qed\]

\begin{proposition}\label{ap:prop:exp}
    $(1 - p) \leq e^{-p}$ for $0 \leq p \leq 1$.
\end{proposition}

\textbf{Proof. } The Taylor series of $e^{-p}$ is alternating with a decreasing sequence, so
\[e^{-p} = 1 - p + \frac{p^2}{2!} - \frac{p^3}{3!} + ... \geq 1 - p. \qed\]

We also give a combinatorial proof of the following result.

\begin{proposition}\label{ap:prop:binom}
    $\binom{2n}{k}^2 \leq \binom{4n}{2k}$ for $n \geq 1$. 
\end{proposition}

\textbf{Proof. } The number of subsets of size $2k$ of a set of size $4n$ is $\binom{4n}{2k}$. This is greater than the number of subsets that can be expressed as the product of two subsets of size $k$ of a set of size $2n$, which is $\binom{2n}{k}^2$. \qed

The proof of the following bound can be found in \cite[Section 6.3]{feller}.
\begin{proposition}\label{ap:prop:rightbinomtail}
    Let $X \sim \mathrm{Bin}(n, p)$. If $r > np$,
    \[ \Pr[X \geq r] \leq \frac{r(1 - p)}{(r - np)^2}.\]
\end{proposition}
Since the binomial distribution is symmetric, we also have the following.
\begin{proposition}\label{ap:prop:leftbinomtail}
    Let $X \sim \mathrm{Bin}(n, p)$. If $r < np$,
    \[ \Pr[X \leq r] \leq \frac{(n - r)p}{(np - r)^2}.\]
\end{proposition}




