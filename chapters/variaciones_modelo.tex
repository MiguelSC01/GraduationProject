\chapter{Variaciones del Modelo}\label{chap:variaciones}

Teniendo ya claros los conceptos médicos y el análisis de un modelo base acorde con tales conceptos, es momento de introducir al modelo las dos enfermedades presentadas en \ref{sec:RBC:enfermedades}. El caso de las hemorragias es más sencillo que el de la anemia, pues los sangrados son tratados una única vez (hasta que sean frenados), mientras que la anemia es una enfermedad cuyo tratamiento es para toda la vida. En este capítulo se presentarán las variaciones matemáticas del modelo y el análisis de las simulaciones de estas, esperando que los resultados obtenidos logren mantener la homeostasis del paciente y, para el caso de la anemia, ver si un aumento o disminución de la dosis afecta realmente los resultados.

\section{Caso con Hemorragia}\label{Sec:variaciones:hemorragia}

Para los sangrados, se tendrán en cuenta diferentes casos. Inicialmente se tendrá en cuenta una hemorragia leve (pérdida del 2$\%$ de sangre) y se observará lo que sucede si, después de la pérdida de sangre, las variables del modelo se mantienen iguales y que sucede si estas se modifican. Posteriormente se considerará el caso de una hemorragia grave (el 14$\%$ de la sangre del cuerpo) y se observarán los efectos de una transfusión sanguínea para recuperar la homeostasis. En el caso de las hemorragias, se tendrá en cuenta el caso en el que $\gamma =1$, pues así el paciente es totalmente sano, el parámetro $f$ se mantiene igual que antes ($=0.00832$) y los valores iniciales también se mantienen.

\subsection{Hemorragia Leve}\label{subsec:variaciones:hemorragia:leve}

Para esta variación del modelo, considere que en el tercer día de medición el paciente, ya sea por un corte o un accidente, pierde el $2\%$ de su cantidad de sangre en el cuerpo que, considerando el caso de que tenga 5 litros de sangre, esto equivale a 100 mililitros de fluido o a $5\times 10^{11}$ glóbulos rojos. Así, el modelo se puede ver de la siguiente manera:

$$R(n+1)= \left\{ \begin{array}{lcc} (1-f)\cdot R(n)+M(n) & \textrm{si} & n \neq 3 \\ \\ (1-f)\cdot R(n)+M(n)-0.02\cdot R(n) & \textrm{si} & n = 3\end{array} \right.$$

$$M(n+1)=\gamma \cdot f \cdot R(n)$$

En donde:
\begin{itemize}
    \item $\gamma=1$;
    \item $f=0.00832$;
    \item $R(0) = 25\times 10^{12};$
    \item $M(0) = 208 \times 10^{9}.$
\end{itemize}

Así, la simulación se ve así: en la primera gráfica de la figura \ref{sec:variaciones:fig:HemoLeveG1} se observa la pérdida de RBC's en el tercer día, un aumento del tercero al cuarto y estabilidad del cuarto en adelante. En la segunda gráfica se observa lo mismo pero con un día de atraso. El aumento que se puede observar está dado por el hecho de que el adendo $M(n)$ conserva lo que ha ocurrido en $R(n-1)$ así, para el cuarto día se producen la cantidad de eritrocitos que se vienen produciendo regularmente antes del accidente, y desde el cuarto día en adelante este adendo ya se acomoda a la nueva cantidad de sangre. La estabilidad que se observa del cuarto día en adelante está dada por el hecho de que $\gamma = 1$, el modelo, de esta manera, funciona como lo hace normalmente (véase \ref{subsec:modelo:simulaciones:G1}) tomando una cantidad inicial de sangre menor que la original, es decir que mantiene una estabilidad con la cantidad de RBC's del cuarto día, que es de $24.504\times 10^{12}$ glóbulos rojos. 

\begin{figure}[H]
    \centering
    \captionsetup{justification=centering}
    \includegraphics[scale=0.534]{figures/HemoLeveG1.png}
    \caption{Simulación del modelo para el caso de una hemorragia sin recuperación de la homeostasis. A la izquierda está la gráfica de $R(n)$, a la derecha la de $M(n)$.}
    \label{sec:variaciones:fig:HemoLeveG1}
\end{figure}

Está claro que esta simulación está totalmente alejada de lo que ocurre en el caso real, pues lo esperado es que el cuerpo recupere con el paso del tiempo la cantidad de glóbulos rojos perdidos, esto quiere decir que para $n>3$ se debe aumentar el valor de $\gamma$ hasta que la cantidad de eritrocitos sea mayor o igual a la inicial, es decir:

$$R(n+1)= \left\{ \begin{array}{lcc} (1-f)\cdot R(n)+M(n) & \textrm{si} & n \neq 3 \\ \\ (1-f)\cdot R(n)+M(n)-0.02\cdot R(n) & \textrm{si} & n = 3\end{array} \right.$$

$$M(n+1)=\left\{ \begin{array}{lcc} f\cdot \gamma_1 \cdot R(n) & \textrm{si} & R(n+1) \geq R(0) \\ \\ f\cdot \gamma_2\cdot R(n) & \textrm{si} & R(n+1)<R(0)\end{array} \right.$$

En donde:
\begin{itemize}
    \item $\gamma_1=1$;
    \item $\gamma_2=1.305$
    \item $f=0.00832$;
    \item $R(0) = 25\times 10^{12};$
    \item $M(0) = 208 \times 10^{9}.$
\end{itemize}

La condición de $M(n+1)$ permite evitar el retraso del que se ha hablado anteriormente, pues si se considera que ya se alcanzó la cantidad "normal" de eritrocitos entonces el cuerpo no debe producir más de los necesarios.

Para estimar el valor de $\gamma_2$ se utilizó la información brindada por el hospital general de Massachusetts en \cite{Massachusetts}, dado que al cuerpo le toma recuperar los glóbulos rojos perdidos en 450 ml de sangre en unas 5 semanas (35 días), entonces 100 ml deben recuperarse en unos 8 días. Las simulaciones hechas muestran que $\gamma=1.305$ permite lograr esta recuperación en ese tiempo.

Para estas simulaciones, se utilizó un tiempo de 15 días para poder observar lo que sucede después de recuperar los eritrocitos perdidos:

\begin{figure}[H]
    \centering
    \captionsetup{justification=centering}
    \includegraphics[scale=0.534]{figures/HemoLeveG13.png}
    \caption{Simulación del modelo para el caso de una hemorragia con recuperación de la homeostasis. A la izquierda está la gráfica de $R(n)$, a la derecha la de $M(n)$.}
    \label{sec:variaciones:fig:HemoLeveG13}
\end{figure}

Lo que se puede ver en estas gráficas es como, a partir del cuarto día (dado el retraso que presenta el modelo) se hace efectivo el cambio de $\gamma_1$ a $\gamma_2$, que se evidencia por el aumento en la primera gráfica de la figura \ref{sec:variaciones:fig:HemoLeveG13} del cuarto al duodécimo día y en la segunda gráfica del cuarto al undécimo día. En el día 12 es en el que por fin se ha superado (o igualado) la cantidad inicial de glóbulos rojos en el cuerpo, el aumento desde la pérdida muestra lo ilustrado en la sección \ref{subsec:modelo:simulaciones:G13}, donde hay un aumento constante de eritrocitos. Después de recuperar la cantidad de sangre perdida, el cuerpo vuelve a estabilizar su sistema, pues ha llegado a la homeostasis. La diferencia en la segunda gráfica de los días 0 a 3 y de los días 13 a 15 se debe a que ha aumentado la cantidad estable de sangre, y para compensar la pérdida el cuerpo debe producir más células.

Teniendo esto en cuenta, ahora se debe analizar el caso en el que el paciente sufre de una hemorragia más grave y debe ser sometido a una transfusión sanguínea para que pueda recuperarse.

\subsection{Hemorragia Grave}

Considérese el caso en el que el paciente analizado sufre una pérdida del 14$\%$ del volumen de sangre de su cuerpo (unos 700 mililitros o 3.25$\times 10^{12}$ eritrocitos) a causa de un accidente, ya sea una hemorragia interna causada por un golpe o una herida externa como un corte. Esta pérdida de sangre es cercana al 20$\%$ que se puede considerar fatal, por lo que la pérdida de sangre presentada se puede considerar como grave y debe ser tratada médicamente mediante una transfusión sanguínea.

Dado que las bolsas de transfusión de glóbulos rojos tienen un volumen aproximado de 250 mililitros (el 5$\%$ de la concentración normal, unos 1.25$\times 10^{12}$ RBC's), entonces para subsanar la pérdida se le suministrarán dos bolsas de eritrocitos y el volumen restante podrá ser recuperado por el cuerpo.

Para poder adecuar el problema al modelo, se considerará que el sangrado inicia y termina en el tercer día y el suministro de sangre ocurre y finaliza en el cuarto. Esta situación no es muy realista, pues en una complicación médica se debe tratar el problema lo más rápido posible, pero para esta investigación es util definir este caso de tal manera para poder ilustrar correctamente las ecuaciones. Es importante notar que el cuerpo tendrá cuatro estados durante este caso: la estabilidad antes y después de la recuperación, el día en el que se hace la transfusión y los días después de la transfusión. De esta manera, el cuerpo debe modificar su producción interna de RBC's en tres momentos, generando dos valores nuevos de $\gamma$.

Así, una hemorragia grave se puede interpretar con la siguiente modificación del modelo base:

$$R(n+1)= \left\{ \begin{array}{lcc} (1-f)\cdot R(n)+M(n) & \textrm{si} & n \neq 3,4 \\ \\ (1-f)\cdot R(n)+M(n)-0.02\cdot R(n) & \textrm{si} & n = 3 \\ \\ (1-f)\cdot R(n)+M(n)+0.1\cdot R(0) & \textrm{si} & n = 4 \\ \end{array} \right.$$

$$M(n+1)=\left\{ \begin{array}{lcc} f\cdot \gamma_1 \cdot R(n) & \textrm{si} & R(n+1) \geq R(0) \\ \\ f\cdot \gamma_2\cdot R(n) & \textrm{si} & n = 4 \\ \\ f\cdot \gamma_3\cdot R(n) & \textrm{si} & n \neq 4\textrm{ y } R(n+1)<R(0)\\ \end{array} \right.$$

En donde:
\begin{itemize}
    \item $\gamma_1=1$;
    \item $\gamma_2=1.331$;
    \item $\gamma_3=1.31$;
    \item $f=0.00832$;
    \item $R(0) = 25\times 10^{12};$
    \item $M(0) = 208 \times 10^{9}.$
\end{itemize}
Los valores de $\gamma_2$, $\gamma_3$ fueron estimados al igual que en el caso de hemorragias leves. La cantidad de glóbulos rojos perdida en 700 mililitros de sangre debería ser recuperada por el cuerpo en unos 55 días, obteniendo un valor de $\gamma_2$ de 1.331. Para estimar $\gamma_3$ se debe tener en cuenta la cantidad de eritrocitos en el cuerpo en el quinto día, pues el cuerpo del paciente se ajusta a la nueva cantidad de glóbulos rojos obtenidos gracias a la transfusión. En el quinto día de simulación, el cuerpo tiene 24.0673$\times 10^{12}$ RBC's, siguiendo la cuenta del hospital general de Massachusetts, la cantidad restante (93.2713 $\times 10^{10}$ eritrocitos) se recuperaría en 15 días, es decir que $\gamma_3 = 1.31$ según las simulaciones del modelo.

Para ilustrar los diferentes estados del cuerpo mencionados anteriormente, se amplió el tiempo de simulación a 22 días: 

\begin{figure}[H]
    \centering
    \captionsetup{justification=centering}
    \includegraphics[scale=0.534]{figures/HemoGrave.png}
    \caption{Simulación del modelo para el caso de una hemorragia grave con transfusión. A la izquierda está la gráfica de $R(n)$, a la derecha la de $M(n)$.}
    \label{sec:variaciones:fig:HemoGrave}
\end{figure}

Para los primeros dos días, el cuerpo se encuentra en el estado estable de $\gamma = 1$. En el tercer día ocurre la hemorragia grave y se puede ver una amplia caída en la cantidad de glóbulos en la gráfica izquierda de la figura \ref{sec:variaciones:fig:HemoGrave}. Esta pérdida, sin embargo, se ve mitigada un poco debido a la producción normal de eritrocitos por parte de la médula ósea. A partir de este día inicia el fuerte crecimiento en la producción de RBC's, pues el cuerpo debe intentar recuperar lo perdido (gráfica derecha a partir del cuarto día). En el día cuatro se puede ver el efecto de la transfusión efectuada a través de un gran aumento en la cantidad de eritrocitos respecto al día anterior. En el quinto hay en la gráfica izquierda una disminución en la pendiente respecto al día anterior que se debe a la transición de $\gamma_2$ a $\gamma_3$, y esto se ve reflejado en la primera gráfica por el cambio de pendiente en los días 4-5 y 5-6. A partir del quinto día el cuerpo produce glóbulos rojos según $\gamma_3$ hasta volver al valor inicial y así volver a la estabilidad brindada por $\gamma_1$.

\section{Caso con Anemia Renal}\label{Sec:variaciones:anemia}

Para este caso se considerará un paciente con anemia renal producida por falta de eritropoyetina en el torrente sanguíneo, es decir que los riñones no segregan la suficiente cantidad de la hormona que le indica al cuerpo que debe acelerar la producción de RBC's.

Como se expuso en la sección \ref{subsec:RBC:enfermedades:anemia}, el tratamiento de inyección de EPO depende del peso del paciente, se debe encontrar un peso en el que basar la dosis pra el modelo. Siguiendo el estudio realizado por el periódico el Tiempo en \cite{elTiempo}, la altura promedio de un hombre colombiano en 2021 es de aproximadamente 172 cm. Utilizando el índice de masa corporal (\cite{IMC}), una medida para hallar el rango de peso ideal según la estatura de una persona, el peso ideal de un hombre que mide 172 cm de altura es de aproximadamente unos 65 kg. De esta manera, la dosis de EPO que se tendrá en cuenta para esta variación del modelo es de 4875 unidades cada tres días (0.975 mU/ml).

Para poder adecuar los parámetros de las ecuaciones, se seguirá el artículo de Panjeta (\cite{panjeta2017interpretation}). Para el grupo de pacientes sin anemia, el promedio de EPO en sangre fue de 11.0 mU/ml, mientras que para el grupo con anemia renal de etapa 4 fue de 8.9 mU/ml. Dado que la eritropoyetina es la hormona encargada de indicarle a la médula ósea que produzca nuevos RBC's, se puede ligar la eritropoyetina al parámetro $\gamma$ del modelo. Dado que este esta dado en una fracción, entonces se tomara que el valor normal $\gamma = 1$ corresponde a la concentración de 11.0 mU/ml de EPO; de esta manera, para los pacientes con anemia renal de cuarta etapa, se tendría que $\gamma = 0.809$. 

El siguiente paso es adecuar las dosis de EPO al modelo. Note que, al considerar que $\gamma$ depende de la concentración de EPO en la sangre, entonces se puede adecuar la ecuación $M(n+1)$ de la siguiente manera:
$$M(n+1)=g(n)\cdot f \cdot R(n),$$

es decir que la cantidad de glóbulos rojos producidos por la médula ósea en el día $n+1$ ahora también dependerá de la concentración de EPO en el día $n$, $g(n)$. Esto implica que se debe definir una ecuación para $g(n)$, esta se basa en el artículo de Frymoyer (\cite{FRYMOYER2019123}), en donde se puede modelar la concentración de un fármaco aplicado por vía intravenosa mediante la ecuación diferencial
$$\dfrac{d\delta}{dt}=-k_e \delta(t),$$
en donde $k_e$ es una constante que representa la velocidad a la que el cuerpo elimina el fármaco. Para el caso de la eritropoyetina, Garzone en \cite{GARZONE2012547}, define $k_e=0.077$. La fracción $\dfrac{d\delta}{dt}$ está medido en [(mU/ml)/h], miliunidades por mililitro por hora.

De esta manera, el modelo para un paciente con anemia renal es:

$$R(n+1)=(1-f)R(n)+M(n),$$
$$M(n+1)=g(n) \cdot f\cdot R(n),$$
$$g(n+1)=\dfrac{\gamma + \delta(24\cdot n)}{\gamma_s},$$
$$\dfrac{d\delta}{dt}=-k_e \delta.$$

En donde:
\begin{itemize}
    \item $\gamma=8.9;$ (concentración EPO paciente) [mU/ml];
    \item $\gamma_s=11;$ (concentración normal EPO) [mU/ml];
    \item $f=0.00832;$
    \item $R(0) = 25\times 10^{12};$
    \item $M(0) = R(0)\cdot f \cdot 0.809 = 145.6\times 10^{9};$
    \item $\delta(0) = 0.975;$ (concentración dosis) [mU/ml]
    \item $k_e=0.077.$ (constante de elminación) [$\textrm{h}^{-1}$]
\end{itemize}

El valor $24\cdot n$ en la función $\delta(t)$ para calcular $g(t)$ se utiliza para acoplar el modelo continuo de la concentración del fármaco al modelo base, que es discreto. El valor $24 \cdot n$ indica la concentración del fármaco al comienzo del día $n$, pues $t$ se mide en horas.

\begin{figure}[H]
    \centering
    \captionsetup{justification=centering}
    \includegraphics[scale=0.534]{figures/AR11.png}
    \includegraphics[scale=0.8]{figures/AR12.png}
    \caption{Simulación del modelo para el caso de anemia renal con dosis de EPO de 0.975 mU/ml. A la izquierda está la gráfica de $R(n)$, a la derecha la de $M(n)$, abajo la de $\delta(t)$.}
    \label{sec:variaciones:fig:Anemia1}
\end{figure}

Como bien se puede ver en la figura \ref{sec:variaciones:fig:Anemia1}, la dosis de EPO administrada no es suficiente para sanar la constante pérdida de glóbulos rojos en el paciente, pues $R(n)$ presenta una disminución en el tiempo. Esto es un resultado esperado, pues $\gamma + \delta(0)=9.875 < \gamma_s$, y dado que la concentración del fármaco se estabiliza rápido, nunca se llega al valor esperado de 11. Sin embargo, la simulación muestra un esfuerzo del cuerpo por aumentar su producción de eritrocitos, como se ve en la gráfica para $M(n)$. De esta manera, para intentar alcanzar una estabilidad en $R(n)$ se debe aumentar la concentración de la dosis.

De esta manera, se realizaron varias simulaciones para hallar una concentración de la dosis que permita una estabilidad en la cantidad de glóbulos rojos del paciente. Se halló que para $\delta(0) = 5.307 [mU/mL]$ se logra llegar a la estabilidad esperada: 
\begin{figure}[H]
    \centering
    \captionsetup{justification=centering}
    \includegraphics[scale=0.526]{figures/AR21.png}
    \includegraphics[scale=0.8]{figures/AR22.png}
    \caption{Simulación del modelo para el caso de anemia renal con dosis de EPO de 5.309mU/mL. A la izquierda está la gráfica de $R(n)$, a la derecha la de $M(n)$, abajo la de $\delta(t)$.}
    \label{sec:variaciones:fig:Anemia2}
\end{figure}

La figura \ref{sec:variaciones:fig:Anemia2} muestra como, gracias a la nueva dosis, el cuerpo es capaz de lograr una oscilación constante para un valor máximo de RBC's ligeramente menor al original. La función $R(n)$ se vuelve periódica con un máximo de $24.9817$ glóbulos rojos, menos del $1\%$ de pérdida respecto al valor original.