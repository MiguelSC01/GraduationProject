\chapter{Introduction}\label{chap:intro}


The Probabilistic Method is a powerful tool, with applications in Combinatorics, Graph Theory, Number Theory and Computer Science. It is a nonconstructive method that proves the existence of an object with a certain property, usually a graph, by showing that the probability that a randomly chosen object has that property is greater than zero. In this case, we will apply the probabilistic method to numerical semigroups. \par

A numerical semigroups is a subset of $\NN$ that is closed under addition (Definition \ref{def:smgps}). These objects are studied in the context of commutative algebra and algebraic geometry, and they have applications in integer programming, coding theory and cryptography \cite{assi2020numerical}. There are numerical invariants that are used to study numerical semigroups, such as the embedding dimension, the genus and the Frobenius number (Definitions \ref{def:smgps:embedding_dim}, \ref{def:smgps:genus} and \ref{def:smgps:frobeniusnum}). For example, the Frobenius number is defined as the maximum of the complement of the numerical semigroup over the integers. \par

The Erdös-Rényi (ER) model is a commonly used model of random graphs, where each edge is chosen with probability $p$, independently of the other edges (Definition \ref{def:probmet:ermodel}). In contrast to graphs, the algebraic nature of numerical semigroups allows for a wider range of random models. This thesis investigates the average behavior random numerical semigroup invariants using a probabilistic model similar to the ER model (Definition \ref{def:randnumsems:ermodel}).


Our central result is Theorem \ref{thm:main}, which is similar to Theorem \ref{thm:ermodel}, the main result of 
\begin{itemize}
    \item \fullcite{de2018random}.
\end{itemize}
Theorem \ref{thm:ermodel} describes the behavior of the expected embedding dimension, genus, and Frobenius number of a random numerical semigroup, depending on the parameters of the model. It gives an explicit bound of the expected value of these invariants. On the other hand, Theorem \ref{thm:main} describes the behavior of the invariants almost surely, that is, with probability that tends to one as the parameters of the model converge to certain values. Our proof is more elementary and provides asymptotically tighter bounds of the behaviour of these invariants. 

We used experiments to study the behavior of ER-type random numerical semigroups, which led to the proof of Theorem \ref{thm:main}. For our experiments, we used \verb|numsgps-sage| \cite[O'Neill]{oneill2018}\cite[Delgado]{delgado2015numericalsgps} and for visualizations we used \verb|IntPic| \cite[Delgado]{delgado2013intpic}. We also implemented our own publicly available repository \verb|randnumsgps| \cite{morales2023} for generating and visualizing random numerical semigroups in Python.\par

The structure of the thesis is as follows:

\begin{itemize}
    \item Chapter \ref{chap:probmet} discusses the Probabilistic Method, based on the work of Noga Alon and Joel H. Spencer.
    \item Chapter \ref{chap:numsems} focuses on numerical semigroups, providing definitions, examples, and results necessary for understanding their structure.
    \item Chapter \ref{chap:randnumsems} introduces three models of random numerical semigroups, including our newly proposed model. We also present recent results in the field, including Theorem \ref{thm:ermodel}. 
    \item Chapter \ref{chap:experiments} details the algorithms and experiments conducted.
    \item Chapter \ref{chap:results} presents the main results, including the proof of Theorem \ref{thm:main}, its implications and its relation with Theorem \ref{thm:ermodel}.
\end{itemize}

The aim of this thesis is to extend the methods used in the study of random graphs to numerical semigroups. The research seeks to contribute to the understanding of random numerical semigroups through a probabilistic perspective. \par
To sum up, we provide a detailed study of random numerical semigroups using probabilistic models, experimental data and software tools.\par
